\documentclass{article}
\usepackage[top=.5in, bottom=.5in, left=.9in, right=.9in]{geometry}
\usepackage[latin1]{inputenc}
\usepackage{enumerate}
\usepackage{hyperref}
\usepackage{graphics}
\usepackage{graphicx}
\usepackage{caption}
\usepackage{subcaption}
\usepackage{tabularx}
\usepackage{amsmath}
\usepackage{amssymb}
\usepackage{siunitx}
\usepackage{mathtools}

\newcommand{\obar}[1]{\ensuremath{\overline{ #1 }}}
\newcommand{\iid}{\ensuremath{\stackrel{\textrm{iid}}{\sim}}}

\usepackage{xcolor}
\definecolor{darkgreen}{rgb}{0,0.25,0}
\newcommand{\soln}{{\color{red}\textbf{Solution:~}\color{black}}}


\usepackage[formats]{listings}
\lstdefineformat{R}{~={\( \sim \)}}
\lstset{% general command to set parameter(s)
basicstyle=\small\ttfamily, % print whole listing small
keywordstyle=\bfseries\rmfamily,
keepspaces=true,
% underlined bold black keywords
commentstyle=\color{darkgreen}, % white comments
stringstyle=\ttfamily, % typewriter type for strings
showstringspaces=false,
numbers=left, numberstyle=\tiny, stepnumber=1, numbersep=5pt, %
frame=shadowbox,
rulesepcolor=\color{black},
,columns=fullflexible,format=R
} %
\renewcommand{\ttdefault}{cmtt}
% enumerate is numbered \begin{enumerate}[(I)] is cap roman in parens
% itemize is bulleted \begin{itemize}
% subfigures:
% \begin{subfigure}[b]{0.5\textwidth} \includegraphics{asdf.jpg} \caption{} \label{subfig:asdf} \end{subfigure}
\hypersetup{colorlinks=true, urlcolor=blue, linkcolor=blue, citecolor=red}


\graphicspath{ {C:/Users/Evan/Desktop/} }
\title{\vspace{-6ex}HW 4\vspace{-2ex}}
\author{Evan Ott \\ UT EID: eao466\vspace{-2ex}}
%\date{DATE}
\setcounter{secnumdepth}{1}
\usepackage[parfill]{parskip}



\begin{document}
\maketitle
\section{Notes from class}
Adding the column of 1s to the $X$ matrix actually shrinks the intercept, since it's going to be penalized as well as
the other $\beta$s.

For the update, typically use the following hack:
\begin{align*}
\hat{\beta}_j = \hat{\beta}_j + 2\lambda \cdot \texttt{skips} \cdot \hat{\beta}_j
\end{align*}
Shouldn't update the Adagrad diagonal matrix, and should also check the sign of the resultant $\hat{\beta}_i$.
In reality, it can't possibly change the sign, but could move it close to zero, so cut off the changes there.

If we don't update the diagonal matrix, then life is easy. We just have 
\begin{align*}
\hat{\beta}_j^{i+1}=\left(1-\lambda \gamma^{(1)}\right)^{i} \hat{\beta}_j^{(1)}
\end{align*}


If you want to use the \texttt{foreach} package, you have to register a parallel backend such as \texttt{doMC}.
It also has errors in the R gui, so it's better usually to run it from the command line.

\end{document}