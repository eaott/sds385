\documentclass{article}
\usepackage[top=.5in, bottom=.5in, left=.9in, right=.9in]{geometry}
\usepackage[latin1]{inputenc}
\usepackage{enumerate}
\usepackage{hyperref}
\usepackage{graphics}
\usepackage{graphicx}
\usepackage{caption}
\usepackage{subcaption}
\usepackage{tabularx}
\usepackage{amsmath}
\usepackage{amssymb}
\usepackage{siunitx}
\usepackage{mathtools}

\usepackage[authoryear,round]{natbib}




\newcommand{\obar}[1]{\ensuremath{\overline{ #1 }}}
\newcommand{\iid}{\ensuremath{\stackrel{\textrm{iid}}{\sim}}}

\newcommand{\op}[2]{{\ensuremath{\underset{ #2 }{\operatorname{ #1 }}~}}}
\newcommand{\norm}[1]{{ \ensuremath{ \left\lVert  #1 \right\rVert  }  }}


\usepackage{xcolor}
\definecolor{darkgreen}{rgb}{0,0.25,0}
\newcommand{\soln}{{\color{red}\textbf{Solution:~}\color{black}}}


\usepackage[formats]{listings}
\lstdefineformat{R}{~={\( \sim \)}}
\lstset{% general command to set parameter(s)
basicstyle=\small\ttfamily, % print whole listing small
keywordstyle=\bfseries\rmfamily,
keepspaces=true,
% underlined bold black keywords
commentstyle=\color{darkgreen}, % white comments
stringstyle=\ttfamily, % typewriter type for strings
showstringspaces=false,
numbers=left, numberstyle=\tiny, stepnumber=1, numbersep=5pt, %
frame=shadowbox,
rulesepcolor=\color{black},
,columns=fullflexible,format=R
} %
\renewcommand{\ttdefault}{cmtt}
% enumerate is numbered \begin{enumerate}[(I)] is cap roman in parens
% itemize is bulleted \begin{itemize}
% subfigures:
% \begin{subfigure}[b]{0.5\textwidth} \includegraphics{asdf.jpg} \caption{} \label{subfig:asdf} \end{subfigure}
\hypersetup{colorlinks=true, urlcolor=blue, linkcolor=blue, citecolor=blue}


\graphicspath{ {C:/Users/Evan/Desktop/} }
\title{\vspace{-6ex}HW 7\vspace{-2ex}}
\author{Evan Ott \\ UT EID: eao466\vspace{-2ex}}
%\date{DATE}
\setcounter{secnumdepth}{0}
\usepackage[parfill]{parskip}



\begin{document}
\maketitle

\section{Laplacian smoothing}

\subsection{(A)}
FIXME: do this


\subsection{(B)}
The problem is
\begin{align*}
\op{minimize}{x\in \mathbb{R}^n} \frac{1}{2}\norm{y-x}_2^2 + \frac{\lambda}{2}x^\top L x=\frac{1}{2}\left(x^\top x - 2y^\top x + y^\top y + \lambda x^\top L x\right)\\
\end{align*}
which we can find a solution for by taking the gradient w.r.t. $x$.
\begin{align*}
0&=\frac{1}{2}\left(2x - 2y + 0 + \lambda (L + L^\top) x\right)\\
&= \left(I + \frac{1}{2}\lambda (L + L^\top)\right)x - y\\
\left(I + \lambda L\right)\hat{x}  &= y
\end{align*}

\subsection{(C)}

\subsubsection{Gauss-Seidel}
Solving $Ax=b$ in this framework amounts to splitting $A=L_*+U$ where $L_*$ is the lower triangular matrix
(including the diagonal) and $U$ is the upper triangular matrix (excluding the diagonal).

The algorithm is then re-writing the problem iteratively as which can be solved with forward substitution.
$$L_* x^+ = b - Ux$$

See \cite{barrett1994templates} \S2.2.2 and Equation (2.6) (slightly different notation but same result).

\subsubsection{Jacobi}
Again solving $Ax=b$, we split into $A=D+R$ where $D$ is just the diagonals and $R$ is everything else (with 0 along the diagonal).

We then iterate
$$x^+ = D^{-1} (b-Rx)$$

What's nice here from an efficiency perspective is that $D$ is easily invertible (so long as there are no zeros in the diagonal), and potentially linearizable depending on the underlying code implementation.

See \cite{barrett1994templates} \S2.2.1 and Equation (2.4) (slightly different notation but same result).


\subsubsection{Conjugate Gradient}
$Ax=b$ is equivalent to minimizing $\phi(x)=\frac{1}{2}x^\top A x -b^\top x$ if $A$ is symmetric.

\begin{align*}
\nabla \phi(x) = Ax-b \equiv r(x)
\end{align*}

Then define $\{ p_0, p_1, \ldots, p_n\}$ as being conjugate w.r.t. $A$, such that
$p_i^\top A p_j = 0$ when $i\neq j$.

Now the algorithm proceeds as
\begin{align*}
x^{(k+1)} &= x^{(k)} + \alpha_k p_k\\
\alpha_k &= -\frac{r_k^\top p_k}{p_k^\top A p_k}
\end{align*}

So now we need to determine how to construct the $p_k$ vectors. You could use eigenvectors, but those
are in general pretty expensive to calculate. So now, let
\begin{align*}
p_0 &= -r_0\\
p_k &= -r_k + \beta_k p_{k-1}\\
\beta_k &= \frac{r_{k+1}^\top r_{k+1}}{r_k^\top r_k}
\end{align*}



\subsection{Notes from class}
Next week: graph fused lasso.

Conjugate gradient is a lot more complicated, and is in fact solving a more general class of problems called Krylov subspace
problems. And it is \emph{fast} if the matrix falls into this
Laplacian class of matrices (certain properties). In that case, as opposed to a standard matrix inverse, $O(n^3)$ or
a sparse one, $O(n^2)$, it will actually be more like $O(n \ln n)$.

It turns out to be important to have a preconditioner. Solving $Ax=b$ is the same as solving $P^{-1}Ax=P^{-1}b$
where $P$ is a ``preconditioner.'' The closer $A$ is to $P$ (while $P$ is still easy to invert), the closer $P^{-1}A$ 
is to the identity, making the problem trivial. The current state of the art is the \emph{algebraic multigrid}
which is that $O(n \ln n)$ type solution.

\vspace{5ex} 

Notation for the rest of the notes:
\begin{align*}
C_\lambda\hat{x}&=y\\
\hat{x}&=C_\lambda^{-1}y
\end{align*}
so $\hat{x}$ is the smoothed/predicted $y$.

The question now is how to choose $\lambda$, potentially using $C_p$ or AIC/BIC, cross-validation, etc.

The leave-one-out lemma (from \cite{trevor2001elements})
allows us to calculate the LOOCV error. Assume $\hat{y}=Sy$
where $y,\hat{y}\in \mathbb{R}^n$ and $S$ is a smoothing matrix (so the linear case we care about).

We need the degrees of freedom of an estimator/model (basically number of free parameters).
\begin{align*}
y&=X\beta + \epsilon\\
\hat{y}&=X\hat{\beta}
\end{align*}
has $p$ degrees of freedom if $\beta\in\mathbb{R}^p$. We need to modify the definition to handle other cases.

Situations we need to address at a minimum:
\begin{enumerate}[1.]
\item Fit to $p$ variables (as an example, OLS)
\item Choose $p$ from $D>p$ candidate variables, then find the best fit for these $p$. This should have more degrees
of freedom if our definition is supposed to make sense at all.
\end{enumerate}

Suppose we have $\hat{y}$ such that $\hat{y}_i=g_i(y)$. Then
\begin{align*}
\textrm{df}(\hat{y}) &= \frac{1}{\sigma^2} \sum_{i=1}^n \textrm{cov}(\hat{y}_i, y_i)\\
\sigma^2&=\textrm{var}(y_i)
\end{align*}

Case: extreme overfitting: $\hat{y}_i=y_i$. Then
\begin{align*}
\textrm{df}(\hat{y}) = \frac{1}{\sigma^2}\sum_{i=1}^n \textrm{var}(y_i)=n
\end{align*}

Case: extreme underfitting: $\hat{y}_i=\obar{y}$ Then
\begin{align*}
\textrm{df}(\hat{y}) &= \frac{1}{\sigma^2}\sum_{i=1}^n \textrm{cov}(\obar{y}, y_i)\\
\textrm{cov}(\obar{y}, y_i)  &=  \textrm{cov}\left( \frac{1}{N}y_i + \frac{1}{N}\sum_{j\neq i}y_j, y_i \right)\\
&=\textrm{cov}\left(\frac{1}{N} y_i, y_i\right)=\frac{1}{N}\sigma^2\\
\textrm{df}(\hat{y})&=\frac{1}{\sigma^2}\sum_{i=1}^n \frac{1}{N}\sigma^2 = 1
\end{align*}

Case: linear smoothers: $\hat{y}=Sy$ then $\textrm{df}(\hat{y}) = \textrm{trace}(S)$.


\section{Graph fused lasso}
Switching to an ADMM framework, we have
\begin{align*}
\op{minimize}{x\in \mathbb{R}^n}&  \frac{1}{2}\norm{y-x}_2^2 + \lambda \norm{r}_1\\
\textrm{subject to}~& Dx-r=0
\end{align*}



\bibliographystyle{plainnat}
\bibliography{../references}


\end{document}