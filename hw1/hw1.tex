\documentclass{article}
\usepackage[top=.5in, bottom=.5in, left=.9in, right=.9in]{geometry}
\usepackage[latin1]{inputenc}
\usepackage{enumerate}
\usepackage{hyperref}
\usepackage{graphics}
\usepackage{graphicx}
\usepackage{caption}
\usepackage{subcaption}
\usepackage{tabularx}
\usepackage{amsmath}
\usepackage{amssymb}
\usepackage{siunitx}
\usepackage{mathtools}

\newcommand{\obar}[1]{\ensuremath{\overline{ #1 }}}
\newcommand{\iid}{\ensuremath{\stackrel{\textrm{iid}}{\sim}}}

\usepackage{xcolor}
\definecolor{darkgreen}{rgb}{0,0.25,0}
\newcommand{\soln}{{\color{red}\textbf{Solution:~}\color{black}}}


\usepackage[formats]{listings}
\lstdefineformat{R}{~={\( \sim \)}}
\lstset{% general command to set parameter(s)
basicstyle=\small\ttfamily, % print whole listing small
keywordstyle=\bfseries\rmfamily,
keepspaces=true,
% underlined bold black keywords
commentstyle=\color{darkgreen}, % white comments
stringstyle=\ttfamily, % typewriter type for strings
showstringspaces=false,
numbers=left, numberstyle=\tiny, stepnumber=1, numbersep=5pt, %
frame=shadowbox,
rulesepcolor=\color{black},
,columns=fullflexible,format=R
} %
\renewcommand{\ttdefault}{cmtt}
% enumerate is numbered \begin{enumerate}[(I)] is cap roman in parens
% itemize is bulleted \begin{itemize}
% subfigures:
% \begin{subfigure}[b]{0.5\textwidth} \includegraphics{asdf.jpg} \caption{} \label{subfig:asdf} \end{subfigure}
\hypersetup{colorlinks=true, urlcolor=blue, linkcolor=blue, citecolor=red}


\graphicspath{ {C:/Users/Evan/Desktop/} }
\title{\vspace{-6ex}SDS385 HW 1\vspace{-2ex}}
\author{Evan Ott \\ UT EID: eao466\vspace{-2ex}}
%\date{DATE}
\setcounter{secnumdepth}{0}
\usepackage[parfill]{parskip}



\begin{document}
\maketitle
\section{Linear Regression}
\subsection{(A)}
WLS objective function:
\begin{align*}
\sum_{i=1}^N\frac{w_i}{2}(y_i-x_i^\top\beta)^2&=\frac{1}{2}\sum_{i=1}^Ny_iw_iy_i - \sum_{i=1}^Ny_iw_ix_i^\top\beta+\frac{1}{2}\sum_{i=1}^Nx_i^\top\beta w_ix_i^\top\beta\\
&=\frac{1}{2}y^\top W y - y^\top W X \beta + \frac{1}{2}(X\beta)^\top WX\beta\\
&=\frac{1}{2}(y-X\beta)^\top W(y-X\beta).
\end{align*}
Minimizing this function means setting the gradient (with respect to $\beta$) to zero:
\begin{align*}
\nabla_\beta\left[\frac{1}{2}(y-X\beta)^\top W(y-X\beta)\right]&=0
\end{align*}
That is
\begin{align*}
\nabla_\beta\left[\frac{1}{2}(y-X\beta)^\top W(y-X\beta)\right]&=0-(y^\top WX)^\top+\frac{2}{2}X^\top WX\hat{\beta}=0\\
&\Rightarrow (X^\top WX)\hat{\beta}=X^\top Wy~~~~_\blacksquare
\end{align*}

\subsection{(B)}
Maybe something with $LDL^\top$ factorization because it's close to that form already, but $X$ is not lower-triangular.

Main idea here is actually not taking the inverse. It's that there are algorithms for solving $(X^\top WX)\hat{\beta}=X^\top Wy$ for $\hat{\beta}$ by treating this as $Ax=b$ and solving for $x$ (to use the traditional variables for the problem).
Good reference for using a Cholesky decomposition is at \url{http://www.seas.ucla.edu/~vandenbe/103/lectures/chol.pdf}




\end{document}