\documentclass{article}
\usepackage[top=.5in, bottom=.5in, left=.9in, right=.9in]{geometry}
\usepackage[latin1]{inputenc}
\usepackage{enumerate}
\usepackage{hyperref}
\usepackage{graphics}
\usepackage{graphicx}
\usepackage{caption}
\usepackage{subcaption}
\usepackage{tabularx}
\usepackage{amsmath}
\usepackage{amssymb}
\usepackage{siunitx}
\usepackage{mathtools}

\usepackage[authoryear,round]{natbib}

\newcommand{\obar}[1]{\ensuremath{\overline{ #1 }}}
\newcommand{\iid}{\ensuremath{\stackrel{\textrm{iid}}{\sim}}}
\newcommand{\op}[2]{{\ensuremath{\underset{ #2 }{\operatorname{ #1 }}~}}}
\newcommand{\norm}[1]{{ \ensuremath{ \left\lVert  #1 \right\rVert  }  }}
\newcommand{\cov}{ \ensuremath{ \textrm{cov} } }
\newcommand{\var}{ \ensuremath{ \textrm{var} } }
\newcommand{\tr}{ \ensuremath{ \textrm{trace} } }
\newcommand{\df}{ \ensuremath{ \textrm{df} } }
\newcommand{\R}{ \ensuremath{ \mathbb{R} }}

\usepackage{xcolor}
\definecolor{darkgreen}{rgb}{0,0.25,0}
\newcommand{\soln}{{\color{red}\textbf{Solution:~}\color{black}}}


\usepackage[formats]{listings}
\lstdefineformat{R}{~={\( \sim \)}}
\lstset{% general command to set parameter(s)
basicstyle=\small\ttfamily, % print whole listing small
keywordstyle=\bfseries\rmfamily,
keepspaces=true,
% underlined bold black keywords
commentstyle=\color{darkgreen}, % white comments
stringstyle=\ttfamily, % typewriter type for strings
showstringspaces=false,
numbers=left, numberstyle=\tiny, stepnumber=1, numbersep=5pt, %
frame=shadowbox,
rulesepcolor=\color{black},
,columns=fullflexible,format=R
} %
\renewcommand{\ttdefault}{cmtt}
% enumerate is numbered \begin{enumerate}[(I)] is cap roman in parens
% itemize is bulleted \begin{itemize}
% subfigures:
% \begin{subfigure}[b]{0.5\textwidth} \includegraphics{asdf.jpg} \caption{} \label{subfig:asdf} \end{subfigure}
\hypersetup{colorlinks=true, urlcolor=blue, linkcolor=blue, citecolor=blue}


\graphicspath{ {C:/Users/Evan/Desktop/} }
\title{\vspace{-6ex}HW 9\vspace{-2ex}}
\author{Evan Ott \\ UT EID: eao466\vspace{-2ex}}
%\date{DATE}
\setcounter{secnumdepth}{0}
\usepackage[parfill]{parskip}



\begin{document}
\maketitle

\section{Notes from class 11/14}
\subsection{PCA}
Idea from PCA is to summarize most of the information in a smaller number of covariates. To do this,
you maximize the variance along the principal axis. This turns out to be basically just an eigenvalue problem,
with the most important axis being the one with the largest eigenvalue, etc. Because of this, scaling of variables
is critical. What's even nicer is that we can find these from:
\begin{align*}
\cov &= X^\top X / (n-1) = V L V^\top\\
X&=USV^\top\\
\cov &= V \frac{S^2}{n-1} V^\top
\end{align*}
For more on PCA, see \cite{james2013introduction}


\subsection{PMD}
In PCA, $V$ is not guaranteed at all to be sparse. 
That's where the new paper comes in, see \cite{witten2009penalized}.
Now, we'll construct $r$ factors by
\begin{align*}
\op{argmin}{\hat{X}\in M(r)}\norm{X-\hat{X}}_F^2 = \sum_{k=1}^r d_k u_k v_k^\top
\end{align*}
where $\norm{\cdots}_F^2$ is the Frobenius norm, the sum of the squares of all matrix elements.

Now, we will set out to solve
\begin{align*}
\op{minimize}{d,u,v} &\frac12 \norm{X-duv^\top}_F^2\\
\textrm{subject to} ~&\norm{u}_2^2=1~~~~\norm{v}_2^2=1\\
&\norm{u}_1\leq c_1~~~~\norm{v}_1\leq c_2
\end{align*}
The trick is that we end up being able to solve for $d$ trivially so that we really only alternate between solving for $u$ and $v$ until convergence.

A small note: if we dropped the $\mathcal{L}_1$ constraint, then update solution for $u$ would be $\frac{Xv}{\norm{Xv}_2}$ because we just want to maximize $u^\top Xv$ (go in same direction, see Theorem 2.1 in the paper), but have unit length (scale by $\mathcal{L}_2$ norm).

When we then include the $\mathcal{L}_1$ constraint, that's where we get the soft-thresholding piece.

Another small note: generating a rank-$k$ approximation will not be guaranteed to reproduce the original rank-$k$ $X$ matrix because this approximation is \emph{sparse}.

\bibliographystyle{plainnat}
\bibliography{/Users/Evan/GitProjects/tex-docs/references}

\end{document}