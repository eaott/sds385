\documentclass{article}
\usepackage[top=.5in, bottom=.5in, left=.9in, right=.9in]{geometry}
\usepackage[latin1]{inputenc}
\usepackage{enumerate}
\usepackage{hyperref}
\usepackage{graphics}
\usepackage{graphicx}
\usepackage{caption}
\usepackage{subcaption}
\usepackage{tabularx}
\usepackage{amsmath}
\usepackage{amssymb}
\usepackage{siunitx}
\usepackage{mathtools}

\usepackage[authoryear,round]{natbib}




\newcommand{\obar}[1]{\ensuremath{\overline{ #1 }}}
\newcommand{\iid}{\ensuremath{\stackrel{\textrm{iid}}{\sim}}}

\usepackage{xcolor}
\definecolor{darkgreen}{rgb}{0,0.25,0}
\newcommand{\soln}{{\color{red}\textbf{Solution:~}\color{black}}}


\usepackage[formats]{listings}
\lstdefineformat{R}{~={\( \sim \)}}
\lstset{% general command to set parameter(s)
basicstyle=\small\ttfamily, % print whole listing small
keywordstyle=\bfseries\rmfamily,
keepspaces=true,
% underlined bold black keywords
commentstyle=\color{darkgreen}, % white comments
stringstyle=\ttfamily, % typewriter type for strings
showstringspaces=false,
numbers=left, numberstyle=\tiny, stepnumber=1, numbersep=5pt, %
frame=shadowbox,
rulesepcolor=\color{black},
,columns=fullflexible,format=R
} %
\renewcommand{\ttdefault}{cmtt}
% enumerate is numbered \begin{enumerate}[(I)] is cap roman in parens
% itemize is bulleted \begin{itemize}
% subfigures:
% \begin{subfigure}[b]{0.5\textwidth} \includegraphics{asdf.jpg} \caption{} \label{subfig:asdf} \end{subfigure}
\hypersetup{colorlinks=true, urlcolor=blue, linkcolor=blue, citecolor=blue}


\graphicspath{ {C:/Users/Evan/Desktop/} }
\title{\vspace{-6ex}HW 7\vspace{-2ex}}
\author{Evan Ott \\ UT EID: eao466\vspace{-2ex}}
%\date{DATE}
\setcounter{secnumdepth}{0}
\usepackage[parfill]{parskip}



\begin{document}
\maketitle

\section{HW}




Here, algorithms are based on \S6.4 in \citep{boyd2011distributed}. For more information, see \S3.3 and \S5 in \citep{boyd2004convex}.



The lasso is commonly written as
\begin{align*}
x^*=\arg\min_x\left(\frac{1}{2}\lVert Ax-b\rVert_2^2 + \lambda \lVert x\rVert_1\right)
\end{align*}
Similar to the last homework, we can split this up into the differentiable part and non-differentiable part, which
gives rise to the ADMM form.
\begin{align*}
\textrm{minimize}~~ &\lVert Ax-b\rVert_2^2 + \lambda \lVert z\rVert_1\\
\textrm{s.t.}~~&x-z=0
\end{align*}
In other words, 
\begin{align*}
f(x)&=\frac{1}{2}\lVert Ax-b\rVert_2^2 = \frac{1}{2}x^\top A^\top A x - b^\top Ax+\frac{1}{2}y^\top y=\frac{1}{2}x^\top Px+q^\top x+r\\
P&=A^\top A\\
q&=-A^\top b\\
g(z)&=\lambda \lVert z\rVert_1\\
Ix + (-I)z &= x-z = c = 0
\end{align*}
So we can break up the minimization according to \S4.2 to find the optimal value for $x$:
\begin{align*}
x^+&=\arg\min_x \left( f(x) + \frac{\rho}{2}\left\lVert x-(z+0-u) \right\rVert_2^2 \right)\\
&=\arg\min_x \left( f(x) + \frac{\rho}{2}\left\lVert x - z + u \right\rVert_2^2 \right)\\
&=\arg\min_x \left( \frac{1}{2}x^\top Px+q^\top x+r + \frac{\rho}{2}\left\lVert x - z + u \right\rVert_2^2 \right)\\
&=\left(P + \rho I^\top I\right)^{-1}\left(\rho I v-q\right)~~~\textrm{[see equation (4.1) for quadratic functions]}\\
&=\left( A^\top A+ \rho I\right)^{-1}\left(\rho  (z - u)+A^\top b\right)
\end{align*}

Now, the update for $z$:
\begin{align*}
g(z)&=\lambda \lVert z\rVert_1=\sum_{i=1}^p \lambda |z_i|=\sum_{i=1}^p h(z_i)
\end{align*}
so the objective for $z$ is component separable, so we can minimize each element separately:
\begin{align*}
\textrm{minimize}~~ &g(z)\\
\textrm{s.t.}~~ & x-z=0\\
z_i^+&=\arg\min_{z_i} \left(h(z_i) + \frac{\rho}{2} (x_i-z_i+u_i)^2 \right)\\
&=\arg\min_{z_i} \left(h(z_i) + \frac{\rho}{2} (z_i-(x_i+u_i))^2 \right)\\
v_i&=x_i+u_i  \\
&=\arg\min_{z_i} \left(h(z_i) + \frac{\rho}{2} (z_i - v_i)^2 \right)\\
&=S_{\lambda/\rho}(v_i)~~~\textrm{[see \S4.4.3]}\\
&=S_{\lambda/\rho}(x_i+u_i)
\end{align*}

Then finally for $u$,
\begin{align*}
u^+&=u+I x + (-I)z-0=u+x-z~~~\textrm{[see equation (3.7)]}
\end{align*}



TODO: refactor code to have ``accelerated proximal gradient'' and ``ADMM'' functions.

See R code on GitHub.

\bibliographystyle{plainnat}
\bibliography{../references}


\end{document}