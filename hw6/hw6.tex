\documentclass{article}
\usepackage[top=.5in, bottom=.5in, left=.9in, right=.9in]{geometry}
\usepackage[latin1]{inputenc}
\usepackage{enumerate}
\usepackage{hyperref}
\usepackage{graphics}
\usepackage{graphicx}
\usepackage{caption}
\usepackage{subcaption}
\usepackage{tabularx}
\usepackage{amsmath}
\usepackage{amssymb}
\usepackage{siunitx}
\usepackage{mathtools}

\newcommand{\obar}[1]{\ensuremath{\overline{ #1 }}}
\newcommand{\iid}{\ensuremath{\stackrel{\textrm{iid}}{\sim}}}

\usepackage{xcolor}
\definecolor{darkgreen}{rgb}{0,0.25,0}
\newcommand{\soln}{{\color{red}\textbf{Solution:~}\color{black}}}


\usepackage[formats]{listings}
\lstdefineformat{R}{~={\( \sim \)}}
\lstset{% general command to set parameter(s)
basicstyle=\small\ttfamily, % print whole listing small
keywordstyle=\bfseries\rmfamily,
keepspaces=true,
% underlined bold black keywords
commentstyle=\color{darkgreen}, % white comments
stringstyle=\ttfamily, % typewriter type for strings
showstringspaces=false,
numbers=left, numberstyle=\tiny, stepnumber=1, numbersep=5pt, %
frame=shadowbox,
rulesepcolor=\color{black},
,columns=fullflexible,format=R
} %
\renewcommand{\ttdefault}{cmtt}
% enumerate is numbered \begin{enumerate}[(I)] is cap roman in parens
% itemize is bulleted \begin{itemize}
% subfigures:
% \begin{subfigure}[b]{0.5\textwidth} \includegraphics{asdf.jpg} \caption{} \label{subfig:asdf} \end{subfigure}
\hypersetup{colorlinks=true, urlcolor=blue, linkcolor=blue, citecolor=red}


\graphicspath{ {C:/Users/Evan/Desktop/} }
\title{\vspace{-6ex}HW 6\vspace{-2ex}}
\author{Evan Ott \\ UT EID: eao466\vspace{-2ex}}
%\date{DATE}
\setcounter{secnumdepth}{0}
\usepackage[parfill]{parskip}



\begin{document}
\maketitle
\section{Proximal operators}
\subsection{(A)}
\begin{align*}
f(x)&\approx \hat{f}(x;x_0)=f(x_0)+(x-x_0)^\top\nabla f(x_0)\\
\textrm{prox}_\gamma \hat{f}(x)&=\arg\min_z \left[ \hat{f}(z) + \frac{1}{2\gamma} \lVert z-x\rVert_2^2 \right]\\
&=\arg\min_z \left[ f(x_0)+(z-x_0)^\top\nabla f(x_0) + \frac{1}{2\gamma} \lVert z-x\rVert_2^2 \right]\\
0&=\frac{\partial}{\partial z}\left[ f(x_0)+(z-x_0)^\top\nabla f(x_0) + \frac{1}{2\gamma} \lVert z-x\rVert_2^2 \right]\\
&=0 + \nabla f(x_0) + \frac{1}{2\gamma}(2z^* - 2x)=\nabla f(x_0) + \frac{1}{\gamma}(z^* - x)\\
\textrm{prox}_\gamma \hat{f}(x)=z^*&=x-\gamma \nabla f(x_0)
\end{align*}
which is indeed the gradient-descent step for $f(x)$ of size $\gamma$ starting at $x_0$.

\subsection{(B)}
\begin{align*}
l(x)&=\frac{1}{2}x^\top P x - q^\top x + r\\
\textrm{prox}_{1/\gamma}l(x)&=\arg\min_z \left[ \hat{l}(z) + \frac{\gamma}{2} \lVert z-x\rVert_2^2 \right]\\
&= \arg\min_z \left[ \frac{1}{2}z^\top P z - q^\top z + r + \frac{\gamma}{2} \lVert z-x\rVert_2^2 \right]\\
0&=\frac{\partial}{\partial z}\left[ \frac{1}{2}z^\top P z - q^\top z + r + \frac{\gamma}{2} \lVert z-x\rVert_2^2 \right]\\
&=Pz^* - q + \gamma (z^* - x)\\
\gamma x + q&= \left(P + \gamma I\right) z^*\\
\textrm{prox}_{1/\gamma}l(x)=z^*&=\left(P + \gamma I\right)^{-1}(\gamma x + q)
\end{align*}
assuming $\left(P + \gamma I\right)^{-1}$ exists.

If we have $ y | x \sim N(Ax, \Omega^{-1})$ with $y$ having $n$ rows, then
\begin{align*}
L(y | x)&=\frac{1}{\sqrt{2\pi}^n} |\Omega|^{-1/2} \exp\left[- \frac{1}{2}(y-Ax)^\top \Omega^{-1} (y-Ax)\right]\\
n(y|x)=-\log L(y|x)&=\frac{1}{2}\log |\Omega| +\frac{n}{2}\log(2\pi) + \frac{1}{2}(y-Ax)^\top \Omega^{-1} (y-Ax)\\
&=\frac{1}{2}y^\top \Omega^{-1} y - (Ax)^\top\Omega^{-1} y + \frac{1}{2}(Ax)^\top\Omega^{-1} Ax + \frac{1}{2}\log |\Omega| +\frac{n}{2}\log(2\pi)
\end{align*}
So $P=\Omega^{-1}$, $q=\Omega^{-1}Ax$ (because $\Omega=\Omega^\top$ since it is a covariance matrix), and $r=\frac{1}{2}(Ax)^\top\Omega^{-1} Ax + \frac{1}{2}\log |\Omega| +\frac{n}{2}\log(2\pi)$.


\subsection{(C)}
\label{sec:1c}
\begin{align*}
\phi(x)&=\tau \lVert x\rVert_1\\
\textrm{prox}_\gamma \phi(x)&=\arg\min_z \left[ \phi(z) + \frac{1}{2\gamma} \lVert z-x\rVert_2^2 \right]\\
&=\arg\min_z \left[ \tau \lVert z \rVert_1 + \frac{1}{2\gamma} \lVert z-x\rVert_2^2 \right]\\
&=\arg\min_z \left[ \tau \sum_{i=1}^n \left(|z_i|\right) + \frac{1}{2\gamma} \sum_{i=1}^n\left( (z_i-x_i)^2 \right) \right]\\
&=\arg\min_z \left[\sum_{i=1}^n \frac{1}{2\gamma}(z_i-x_i)^2 + \tau |z_i| \right]\\
&=\arg\min_z \left[\sum_{i=1}^n \frac{1}{2}(z_i-x_i)^2 + \tau\gamma |z_i| \right]~~~~{\small\textrm{(multiplying by positive scalar yields same optimization)}}
\end{align*}
The term being minimized for each component $z_i$ is exactly $S_{\tau\gamma}(x_i)$ from the notation last week, and
there are no interaction terms between the $z_i$ and $z_j$ for $i\neq j$, so
\begin{align*}
\left(\textrm{prox}_\gamma \phi(x)\right)_i=S_{\tau\gamma}(x_i)
\end{align*}


\section{The proximal gradient method}
\subsection{(A)}
\begin{align*}
\hat{x}&=\arg\min_x \left\{ \tilde{l}(x;x_0)+\phi(x) \right\}\\
&=\arg\min_x \left\{ l(x_0) + (x-x_0)^\top \nabla l(x_0) + \frac{1}{2\gamma} \lVert x-x_0 \rVert_2^2+\phi(x) \right\}\\
&=\arg\min_z \left\{ l(x_0) + (z-x_0)^\top \nabla l(x_0) + \frac{1}{2\gamma} \lVert z-x_0 \rVert_2^2+\phi(z) \right\}\\
&=\arg\min_z \left\{ \phi(z) + l(x_0) + (z-x_0)^\top \nabla l(x_0) + \frac{1}{2\gamma} \left(z^\top z - 2x_0^\top z + x_0^\top x_0 \right) \right\}\\
&=\arg\min_z \left\{ \phi(z) + (z-x_0)^\top \nabla l(x_0) + \frac{1}{2\gamma} \left(z^\top z - 2x_0^\top z + x_0^\top x_0 \right) \right\}~~~~{\small\textrm{(add/subtract a constant for same optimization)}}\\
&=\arg\min_z \left\{ \phi(z) + \frac{\gamma}{2}\left[\nabla l(x_0)\right]^\top\nabla l(x_0)+2\frac{1}{2\gamma}(z-x_0)^\top \gamma\nabla l(x_0) + \frac{1}{2\gamma} \left(z^\top z - 2x_0^\top z + x_0^\top x_0 \right) \right\}\\
&=\arg\min_z \left\{ \phi(z) + \frac{1}{2\gamma}\left[\gamma\nabla l(x_0)\right]^\top\gamma\nabla l(x_0)+2\frac{1}{2\gamma}(z-x_0)^\top \gamma\nabla l(x_0) + \frac{1}{2\gamma} \left(z^\top z - 2x_0^\top z + x_0^\top x_0 \right) \right\}\\
&=\arg\min_z \left\{ \phi(z) + \frac{1}{2\gamma}\left(\left[\gamma\nabla l(x_0)\right]^\top\gamma\nabla l(x_0)+2(z-x_0)^\top \gamma\nabla l(x_0) +  z^\top z - 2x_0^\top z + x_0^\top x_0 \right) \right\}\\
&=\arg\min_z \left[\phi(z) + \frac{1}{2\gamma} \lVert z - x_0 + \gamma\nabla l(x_0) \rVert_2^2\right]\\
&=\arg\min_z \left[\phi(z) + \frac{1}{2\gamma} \lVert z - (x_0 - \gamma\nabla l(x_0)) \rVert_2^2\right]\\
u&=x_0-\gamma\nabla l(x_0)\\
\hat{x}&=\textrm{prox}_\gamma \phi(u)\\
\end{align*}



\subsection{(B)}
Now, we want to play around with our results to cast the lasso regression into a proximal gradient problem.
\begin{align*}
\hat{\beta}&=\arg\min_\beta \left\{\lVert y-X\beta\rVert_2^2 + \lambda \lVert \beta\rVert_1\right\}\\
l(\beta | X,y)&=\lVert y-X\beta\rVert_2^2=y^\top y - 2y^\top X\beta + \beta^\top X^\top X \beta\\
\hat{\beta}&=\arg\min_\beta \left\{l(\beta | X,y)+ \lambda \lVert \beta\rVert_1\right\}\\
l(\beta | X,y)\approx\hat{l}(\beta | X,y ; \beta_0)&= l(\beta_0 | X, y) + (\beta-\beta_0)^\top \nabla l(\beta_0 | X, y)\\
\nabla l(\beta | X, y)&=0 -2X^\top y + 2X^\top X\beta\\
\hat{l}(\beta | X,y ; \beta_0)&=\lVert y-X\beta_0\rVert_2^2 + (\beta-\beta_0)^\top \left(-2X^\top y + 2X^\top X\beta_0\right)
\end{align*}
Now, in the linear approximation to $l(\beta | X,y)$, we add in the regularization:
\begin{align*}
\tilde{l}(\beta | X,y ; \beta_0)&=\lVert y-X\beta_0\rVert_2^2 + (\beta-\beta_0)^\top \left(-2X^\top y + 2X^\top X\beta_0\right)+\frac{1}{2\gamma}\lVert \beta-\beta_0\rVert_2^2
\end{align*}
Now, we let $l(\beta|X, y)\approx \tilde l(\beta | X,y ;\beta_0)$ when $\beta$ is near $\beta_0$.
This is now exactly the form of surrogate optimization referenced above so
\begin{align*}
\phi(\beta)&=\lambda \lVert \beta \rVert_1\\
u^{(t)}&=\beta^{(t)}-\gamma^{(t)} \nabla l(\beta^{(t)} | X, y)= \beta^{(t)} - \gamma^{(t)}\left(2X^\top X\beta^{(t)}-2X^\top y\right)\\
\beta^{(t+1)}&=\textrm{prox}_{\gamma^{(t)}} \phi(u^{(t)})\\
\beta_i^{(t+1)}&=S_{\lambda\gamma^{(t)}}\left(u_i^{(t)}\right)=\textrm{sign}\left(u_i^{(t)}\right)\left(\left|u_i^{(t)}\right|-\lambda\gamma^{(t)}\right)_+
\end{align*}
So, to go from step $t$ to step $t+1$, we just compute $u^{(t)}$ then use its components to compute $\beta_i^{(t+1)}$.

There's a relatively high one-time cost to compute $X^\top X$ and $X^\top y$, and (depending on how big $p$, the number of elements of $\beta$, is) this cost carries over each iteration to compute $X^\top X \beta^{(t)}$. That's a $O(p^2)$ calculation (at least in the dense case). Beyond that, the rest of the operations are $O(p)$.




\section{Notes from class Oct. 17}
Looking today at dual descent, which is the minimal pre-requisite to understand ADMM (hw 7). 

\subsection{Standard-form convex optimization problem}
Note: $x$ will be what we're optimizing.

Minimize $f_0(x)$ subject to $f_i(x) \leq 0$ for $i=1,\ldots, m$ and $Ax=b$ (affine), with $f_0$ and all $f_i$ being convex.

Convex set is geometric: take two points in the set, any point on the line between
them is also in the set. Convex function is similar: look at the affine transformation (I think linear approximation at a point) is a global under-estimator or not.

\subsubsection{Linear program (LP)}
Minimize $c^\top x + d$ subject to $Gx\preceq h$ and $Ax=b$ ($\preceq$ means pointwise inequality [applies the $\leq$ operator element-wise].

\subsubsection{Quadratic program (QP)}
Minimize $\frac{1}{2}x^\top P x + q^\top x + r$ subject to $Gx\preceq h$ and $Ax=b$.

For example, constrained least squares:

minimize $\frac{1}{2}\lVert Ax-b\rVert_2^2$ (which is the optimization way of writing $X\beta-y$), constrained by $l\preceq x \preceq u$. That is, $x\preceq u$ and $-x\preceq l$ which is an example of a QP ($G$ would be a block matrix with identity and negative identity).

\subsection{Slack variables}
Similar in idea to latent variables used in MCMC that augment the model, put it
in a bigger space, and rewrite the problem.

Example: $$\underset{x\in \mathbb{R}^D}{\operatorname{minimize}}~\frac{1}{2}\lVert Ax-b\rVert_2^2 + \lambda \lVert x\rVert_1$$
We rewrite it in as
\begin{align*}
\underset{x\in \mathbb{R}^D,~z\in\mathbb{R}^D}{\operatorname{minimize}}~&\frac{1}{2}\lVert Ax-b\rVert_2^2 + \lambda \lVert z\rVert_1\\
\textrm{subject to}&~x=z
\end{align*}

Example: $$\underset{x\in \mathbb{R}^D}{\operatorname{minimize}}~\frac{1}{2}\lVert x-y\rVert_2^2 + \lambda \lVert Dx\rVert_1$$ where $D$ is an ``oriented edge matrix'' (see spatial smoothing). Can rewrite as
\begin{align*}
\underset{x\in \mathbb{R}^D,~z\in\mathbb{R}^D}{\operatorname{minimize}}~&\frac{1}{2}\lVert x-y\rVert_2^2 + \lambda \lVert z\rVert_1\\
\textrm{subject to}&~Dx=z~~{\small \textrm{feasibility constraint}}
\end{align*}
Often, most algorithms don't enforce the feasibility constraint until convergence.

\subsection{Lagrangian}
Minimize $f_0(x)$ subject to $f_i(x)=\leq 0$ and $h_i(x)=0$ (not necessarily convex). Would like to cast this into an unconstrained optimization.

Define \begin{align*}
I\_(u)&=\left\{\begin{array}{cc}0 & u\leq 0\\\infty & \textrm{o.w.}\end{array}\right.\\
I_0(u)&=\left\{\begin{array}{cc}0 & u= 0\\\infty & \textrm{o.w.}\end{array}\right.
\end{align*}

So now
\begin{align*}
\underset{x\in \mathbb{R}^D}{\operatorname{minimize}}
f_0(x)+\sum_{i=1}^m I\_(f_i(x)) + \sum_{i=1}^p I_0(h_i(x))
\end{align*}
in other words, any time we're in a case where the constraint is not satisfied, our objective jumps to $\infty$.

The Lagrangian just linearizes $I\_(u)$ and $I_0(u)$:

\begin{align*}
L(x, \lambda, nu)&=f_0(x) +\sum_{i=1}^m \lambda_if_i(x) + \sum_{i=1}^p \nu_ih_i(x)
\end{align*}
where $\lambda_i$ and $\nu_i$ are the Lagrange multipliers (also called dual variables). $L(x, \lambda, \nu)$ is the ``primal variable,'' the thing we actually care about.

Now, let's look at the (Lagrange) dual function:
\begin{align*}
g(\lambda, \nu)=\inf_x L(x, \lambda, \nu)
\end{align*}
Why is this useful?

\subsubsection{Fact}
For any $\lambda \succeq 0, \nu$, we have
$g(\lambda, \nu)\leq p^*$ which is the optimal value of the primal problem ($f_0(x^*)$). 

\subsubsection{Proof}

For any $\lambda \succeq 0, \nu$, we have the following.
$$\sum_{i=1}^m \lambda_i f_i(\tilde{x}) + \sum_{i=1}^p\nu_i h_i(\tilde{x}) \leq 0$$
for any feasibile $\tilde{x}$ (all the $h_i$ are 0, all the $f_i \leq 0$ so this has to be true.

Therefore $L(\tilde{x}, \lambda, \nu)=f_0(\tilde{x}) + \sum_{i=1}^m \lambda_i f_i(\tilde{x}) + \sum_{i=1}^p\nu_i h_i(\tilde{x})\leq f_0(\tilde{x})$. And
$$g(\lambda, \nu)=\inf_x L(x, \lambda, \nu)\leq L(\tilde{x}, \lambda, \nu)\leq f_0(\tilde{x})$$
this works for any feasible $\tilde{x}$ so it must be true for the optimal value $x^*$.

\subsubsection{Dual problem}
$$\underset{\lambda\succeq 0,~ \nu\in\mathbb{R}^p}{\operatorname{maximize}}~ g(\lambda, \nu)$$
This is essentially a minimax problem: we're maximizing the lower bound. Let's say that the optimal value is $d^*=g(\lambda^*,~\nu^*)$.

Cool thing: strong (Lagrangian) duality is that $p^*=d^*$ which is kind of incredible, and this
is actually true sometimes. When? That's the \$1,000,000 question for math careers. However, in stats, it's true in basically all interesting convex problems (mostly). More particularly, under Slater's conditions (see Boyd \S5.2).

Now: slight change of notation to match the paper rather than matching the textbook. Dual variables $\lambda, \nu$ will now be denoted $y^*$.

If strong duality holds, then
$$x^*=\arg\min_x L(x, y^*)$$ where $y^*$ is a dual optimal solution (assuming that $L$ has one minimum). Why do we care? Sometimes it's easier to solve the dual problem than solving the primal problem.

\subsection{Dual ascent}
Now, let's assume that these conditions all hold (strong duality, one minimum, Slater's conditions).

{\textbf{Dual ascent}}: solve the dual problem by gradient ascent.

$y$ is the dual variable.
\subsubsection{Example}
minimize $f(x)$ subject to $Ax=b$. What's the lagrangian? Well, $Ax-b=0$ so
$$L(x, y)=f(x)+y^\top (Ax-b)$$

Dual ascent here is $y^{t+1}=y^t + \alpha^t \nabla g(y)$ where $g$ is the dual function $g(y)=\inf_x L(x, y)$.

How do we evaluate the gradient of the dual function? Think it's called the envelope formula (this is a general property of functions, with some regularities). For $g(y)=\inf_x L(x, y)$, we have:
$$\nabla g(y)=\left.\nabla_yL(x,y)\right|_{x=\hat{x}(y)}$$ where $\hat{x}(y)=\arg\min_x L(x, y)$

So in this case,
\begin{align*}
\nabla g(y)&=\nabla_y \left[f(x) + y^\top(Ax-b)\right]\\
&=Ax-b
\end{align*}
which is exactly the residuals of the feasibility constraints. So when $\nabla g(y)=0$ this gives the extremely interpretable result of having a solution when all constraints are met.

So dual ascent becomes:
\begin{align*}
x^{(t+1)}&=\arg\min_x L(x, y^{(t)})\\
y^{(t+1)}&=y^{(t)} + \alpha^{(t)} \left(Ax^{(t+1)}-b\right)
\end{align*}
which is nice because we never actually have to use the dual function. At convergence, $x^{(T)}=x^*$ and $y^{(T)}=y^*$.

This is most of the understanding we need for ADMM, but leaves out the method of multipliers (related to augmented Lagrangian).

\section{Notes from class October 19}

Final project: could be diving into a topic we've covered, applying an algorithm to new, richer data, etc. Could replicate
results of a paper. Should have a non-trivial computational (likely) or theoretical (rare) component. Can work in pairs
if doing a ``new'' project. Should have a 1-2 page outline of an idea by November. Final project should be in \LaTeX or
a python notebook or Rmarkdown.

\section{Notes from class October 24}
HW 8 will be looking at spatial smoothing of fMRI data. This is more or less de-noising as a pre-processing step in terms of fMRI analysis.
Will use Laplacian smoothing ($y$ data, $x$ underlying graph, and $L$ the Laplacian matrix)
in closed form (will need to solve sparse linear system using either existing algorithms we've looked at or an iterative solution and compare).
Will also use the graph fused lasso (not a sparse solution, but piecewise constant so the differences are sparse) which will use ADMM. There's
simple and slow ADMM, or a more efficient one, or the state-of-the-art (milliseconds versus seconds).

Now, back to looking at proximal gradient.

TODO Should compare my own code to glmnet to see how the coefficients did,
ensure that the right lambda conversion is taking place.



















\section{Connection to readings course}
Fun fact that I worked on. Instead of looking at the proximal operator of $|x|$, we can look at the envelope function.
In \nameref{sec:1c} above, let $\tau=n=1$, in other words, letting $\phi(x)=\lVert x\rVert_1=|x|$. I already showed that the
proximal operator is $\textrm{prox}_\gamma \phi(x)=S_\gamma(x)=\textrm{sign}(x)\left(|x|-\gamma\right)_+$.

That's the $\arg\min$ of the regularized objective. So, we can plug that into the regularized objective and we get
the envelope:
\begin{align*}
E_\gamma \phi(x)=E_\gamma |x|&=\left|S_\gamma(x)\right|+\frac{1}{2\gamma}\left\lVert S_\gamma(x) - x \right\rVert_2^2\\
&=\left|\textrm{sign}(x)\left(|x|-\gamma\right)_+\right|+\frac{1}{2\gamma}\left\lVert S_\gamma(x) - x \right\rVert_2^2\\
&=\left(|x|-\gamma\right)_+ + \frac{1}{2\gamma}\left\lVert S_\gamma(x) - x \right\rVert_2^2\\
S_\gamma(x)&=\left\{\begin{array}{cc}
x-\gamma & |x| \geq \gamma \wedge x > 0\\
x + \gamma & |x| \geq \gamma \wedge x < 0\\
0 & \textrm{otherwise}
\end{array}\right.~~~~~{\small\textrm{(reminder from hw5)}}\\
E_\gamma |x|&=\left(|x|-\gamma\right)_+ + \frac{1}{2\gamma}\left\lVert S_\gamma^*(x) \right\rVert_2^2\\
S_\gamma^*(x)&=\left\{\begin{array}{cc}
-\gamma & |x| \geq \gamma \wedge x > 0\\
+ \gamma & |x| \geq \gamma \wedge x < 0\\
-x & \textrm{otherwise}
\end{array}\right.\\
E_\gamma |x|&=\left(|x|-\gamma\right)_+ + \frac{1}{2\gamma} \gamma^2 \cdot I(|x|\geq\gamma) + \frac{1}{2\gamma} x^2 \cdot I(|x|<\gamma)\\
E_\gamma |x|&=\left(|x|-\gamma\right)\cdot I(|x|\geq\gamma) + \frac{\gamma}{2} \cdot I(|x|\geq\gamma) + \frac{1}{2\gamma} x^2 \cdot I(|x|<\gamma)\\
&=\left\{\begin{array}{cc}
|x|-\gamma+\frac{\gamma}{2} & |x| \geq \gamma\\
\frac{x^2}{2\gamma} & \textrm{otherwise}
\end{array}\right.=\left\{\begin{array}{cc}
|x|-\frac{\gamma}{2} & |x| \geq \gamma\\
\frac{x^2}{2\gamma} & \textrm{otherwise}
\end{array}\right.\\
&=\frac{1}{\gamma} H_\gamma(x)
\end{align*}
where $H_\gamma(x)$ is the Huber loss function: $H_\gamma(x) = \frac{1}{2}x^2 \cdot I(|x|<\gamma) + \left(\gamma|x|-\frac{1}{2}\gamma^2\right)\cdot I(|x|\geq \gamma)$

If we generalize a bit to $\phi(x)=\tau |x|$,
\begin{align*}
E_\gamma \phi(x)=E_\gamma \tau|x|&=\left|S_{\tau\gamma}(x)\right|+\frac{1}{2\gamma}\left\lVert S_{\tau\gamma}(x) - x \right\rVert_2^2\\
&=\left|\textrm{sign}(x)\left(|x|-\tau\gamma\right)_+\right|+\frac{1}{2\gamma}\left\lVert S_{\tau\gamma}(x) - x \right\rVert_2^2\\
&=\left(|x|-\tau\gamma\right)_+ + \frac{1}{2\gamma}\left\lVert S_{\tau\gamma}(x) - x \right\rVert_2^2\\
&=\left(|x|-\tau\gamma\right)_+ + \frac{1}{2\gamma}\left\lVert S_{\tau\gamma}^*(x) \right\rVert_2^2\\
S_{\tau\gamma}^*(x)&=\left\{\begin{array}{cc}
-\tau\gamma & |x| \geq \tau\gamma \wedge x > 0\\
+ \tau\gamma & |x| \geq \tau\gamma \wedge x < 0\\
-x & \textrm{otherwise}
\end{array}\right.\\
E_\gamma \tau|x|&=\left(|x|-\tau\gamma\right)_+ + \frac{1}{2\gamma} \tau^2\gamma^2 \cdot I(|x|\geq\tau\gamma) + \frac{1}{2\gamma} x^2 \cdot I(|x|<\tau\gamma)\\
&=\left(|x|-\tau\gamma\right)\cdot I(|x|\geq\tau\gamma) + \frac{\tau^2\gamma}{2} \cdot I(|x|\geq\gamma) + \frac{1}{2\gamma} x^2 \cdot I(|x|<\tau\gamma)\\
&=\left\{\begin{array}{cc}
|x|-\tau\gamma+\frac{\tau^2\gamma}{2} & |x| \geq \tau\gamma\\
\frac{x^2}{2\gamma} & \textrm{otherwise}
\end{array}\right.
\end{align*}
which is at least not obviously directly a function of the Huber loss.




\end{document}