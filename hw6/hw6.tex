\documentclass{article}
\usepackage[top=.5in, bottom=.5in, left=.9in, right=.9in]{geometry}
\usepackage[latin1]{inputenc}
\usepackage{enumerate}
\usepackage{hyperref}
\usepackage{graphics}
\usepackage{graphicx}
\usepackage{caption}
\usepackage{subcaption}
\usepackage{tabularx}
\usepackage{amsmath}
\usepackage{amssymb}
\usepackage{siunitx}
\usepackage{mathtools}

\newcommand{\obar}[1]{\ensuremath{\overline{ #1 }}}
\newcommand{\iid}{\ensuremath{\stackrel{\textrm{iid}}{\sim}}}

\usepackage{xcolor}
\definecolor{darkgreen}{rgb}{0,0.25,0}
\newcommand{\soln}{{\color{red}\textbf{Solution:~}\color{black}}}


\usepackage[formats]{listings}
\lstdefineformat{R}{~={\( \sim \)}}
\lstset{% general command to set parameter(s)
basicstyle=\small\ttfamily, % print whole listing small
keywordstyle=\bfseries\rmfamily,
keepspaces=true,
% underlined bold black keywords
commentstyle=\color{darkgreen}, % white comments
stringstyle=\ttfamily, % typewriter type for strings
showstringspaces=false,
numbers=left, numberstyle=\tiny, stepnumber=1, numbersep=5pt, %
frame=shadowbox,
rulesepcolor=\color{black},
,columns=fullflexible,format=R
} %
\renewcommand{\ttdefault}{cmtt}
% enumerate is numbered \begin{enumerate}[(I)] is cap roman in parens
% itemize is bulleted \begin{itemize}
% subfigures:
% \begin{subfigure}[b]{0.5\textwidth} \includegraphics{asdf.jpg} \caption{} \label{subfig:asdf} \end{subfigure}
\hypersetup{colorlinks=true, urlcolor=blue, linkcolor=blue, citecolor=red}


\graphicspath{ {C:/Users/Evan/Desktop/} }
\title{\vspace{-6ex}HW 6\vspace{-2ex}}
\author{Evan Ott \\ UT EID: eao466\vspace{-2ex}}
%\date{DATE}
\setcounter{secnumdepth}{1}
\usepackage[parfill]{parskip}



\begin{document}
\maketitle
\section{Proximal operators}
\subsection{(A)}
\begin{align*}
f(x)&\approx \hat{f}(x;x_0)=f(x_0)+(x-x_0)^\top\nabla f(x_0)\\
\textrm{prox}_\gamma \hat{f}(x)&=\arg\min_z \left[ \hat{f}(z) + \frac{1}{2\gamma} \lVert z-x\rVert_2^2 \right]\\
&=\arg\min_z \left[ f(x_0)+(z-x_0)^\top\nabla f(x_0) + \frac{1}{2\gamma} \lVert z-x\rVert_2^2 \right]\\
0&=\frac{\partial}{\partial z}\left[ f(x_0)+(z-x_0)^\top\nabla f(x_0) + \frac{1}{2\gamma} \lVert z-x\rVert_2^2 \right]\\
&=0 + \nabla f(x_0) + \frac{1}{2\gamma}(2z^* - 2x)=\nabla f(x_0) + \frac{1}{\gamma}(z^* - x)\\
\textrm{prox}_\gamma \hat{f}(x)=z^*&=x-\gamma \nabla f(x_0)
\end{align*}
which is indeed the gradient-descent step for $f(x)$ of size $\gamma$ starting at $x_0$.

\subsection{(B)}
\begin{align*}
l(x)&=\frac{1}{2}x^\top P x - q^\top x + r\\
\textrm{prox}_{1/\gamma}l(x)&=\arg\min_z \left[ \hat{l}(z) + \frac{\gamma}{2} \lVert z-x\rVert_2^2 \right]\\
&= \arg\min_z \left[ \frac{1}{2}z^\top P z - q^\top z + r + \frac{\gamma}{2} \lVert z-x\rVert_2^2 \right]\\
0&=\frac{\partial}{\partial z}\left[ \frac{1}{2}z^\top P z - q^\top z + r + \frac{\gamma}{2} \lVert z-x\rVert_2^2 \right]\\
&=Pz^* - q + \gamma (z^* - x)\\
\gamma x + q&= \left(P + \gamma I\right) z^*\\
\textrm{prox}_{1/\gamma}l(x)=z^*&=\left(P + \gamma I\right)^{-1}(\gamma x + q)
\end{align*}
assuming $\left(P + \gamma I\right)^{-1}$ exists.

If we have $ y | x \sim N(Ax, \Omega^{-1})$ with $y$ having $n$ rows, then
\begin{align*}
L(y | x)&=\frac{1}{\sqrt{2\pi}^n} |\Omega|^{-1/2} \exp\left[- \frac{1}{2}(y-Ax)^\top \Omega^{-1} (y-Ax)\right]\\
n(y|x)=-\log L(y|x)&=\frac{1}{2}\log |\Omega| +\frac{n}{2}\log(2\pi) + \frac{1}{2}(y-Ax)^\top \Omega^{-1} (y-Ax)\\
&=\frac{1}{2}y^\top \Omega^{-1} y - (Ax)^\top\Omega^{-1} y + \frac{1}{2}(Ax)^\top\Omega^{-1} Ax + \frac{1}{2}\log |\Omega| +\frac{n}{2}\log(2\pi)
\end{align*}
So $P=\Omega^{-1}$, $q=\Omega^{-1}Ax$ (because $\Omega=\Omega^\top$ since it is a covariance matrix), and $r=\frac{1}{2}(Ax)^\top\Omega^{-1} Ax + \frac{1}{2}\log |\Omega| +\frac{n}{2}\log(2\pi)$.


\subsection{(C)}
\begin{align*}
\phi(x)&=\tau \lVert x\rVert_1\\
\textrm{prox}_\gamma \phi(x)&=\arg\min_z \left[ \phi(z) + \frac{1}{2\gamma} \lVert z-x\rVert_2^2 \right]\\
&=\arg\min_z \left[ \tau \lVert z \rVert_1 + \frac{1}{2\gamma} \lVert z-x\rVert_2^2 \right]\\
&=\arg\min_z \left[ \tau \sum_{i=1}^n \left(|z_i|\right) + \frac{1}{2\gamma} \sum_{i=1}^n\left( (z_i-x_i)^2 \right) \right]\\
&=\arg\min_z \left[\sum_{i=1}^n \frac{1}{2\gamma}(z_i-x_i)^2 + \tau |z_i| \right]\\
&=\arg\min_z \left[\sum_{i=1}^n \frac{1}{2}(z_i-x_i)^2 + \tau\gamma |z_i| \right]~~~~{\small\textrm{(multiplying by positive scalar yields same optimization)}}
\end{align*}
The term being minimized for each component $z_i$ is exactly $S_{\tau\gamma}(x_i)$ from the notation last week, and
there are no interaction terms between the $z_i$ and $z_j$ for $i\neq j$, so
\begin{align*}
\left(\textrm{prox}_\gamma \phi(x)\right)_i=S_{\tau\gamma}(x_i)
\end{align*}


\section{The proximal gradient method}
\subsection{(A)}
\begin{align*}
\hat{x}&=\arg\min_x \left\{ \tilde{l}(x;x_0)+\phi(x) \right\}\\
&=\arg\min_x \left\{ l(x_0) + (x-x_0)^\top \nabla l(x_0) + \frac{1}{2\gamma} \lVert x-x_0 \rVert_2^2+\phi(x) \right\}\\
&=\arg\min_z \left\{ l(x_0) + (z-x_0)^\top \nabla l(x_0) + \frac{1}{2\gamma} \lVert z-x_0 \rVert_2^2+\phi(z) \right\}\\
&=\arg\min_z \left\{ \phi(z) + l(x_0) + (z-x_0)^\top \nabla l(x_0) + \frac{1}{2\gamma} \left(z^\top z - 2x_0^\top z + x_0^\top x_0 \right) \right\}\\
&=\arg\min_z \left\{ \phi(z) + (z-x_0)^\top \nabla l(x_0) + \frac{1}{2\gamma} \left(z^\top z - 2x_0^\top z + x_0^\top x_0 \right) \right\}~~~~{\small\textrm{(add/subtract a constant for same optimization)}}\\
&=\arg\min_z \left\{ \phi(z) + \frac{\gamma}{2}\left[\nabla l(x_0)\right]^\top\nabla l(x_0)+2\frac{1}{2\gamma}(z-x_0)^\top \gamma\nabla l(x_0) + \frac{1}{2\gamma} \left(z^\top z - 2x_0^\top z + x_0^\top x_0 \right) \right\}\\
&=\arg\min_z \left\{ \phi(z) + \frac{1}{2\gamma}\left[\gamma\nabla l(x_0)\right]^\top\gamma\nabla l(x_0)+2\frac{1}{2\gamma}(z-x_0)^\top \gamma\nabla l(x_0) + \frac{1}{2\gamma} \left(z^\top z - 2x_0^\top z + x_0^\top x_0 \right) \right\}\\
&=\arg\min_z \left\{ \phi(z) + \frac{1}{2\gamma}\left(\left[\gamma\nabla l(x_0)\right]^\top\gamma\nabla l(x_0)+2(z-x_0)^\top \gamma\nabla l(x_0) +  z^\top z - 2x_0^\top z + x_0^\top x_0 \right) \right\}\\
&=\arg\min_z \left[\phi(z) + \frac{1}{2\gamma} \lVert z - x_0 + \gamma\nabla l(x_0) \rVert_2^2\right]\\
&=\arg\min_z \left[\phi(z) + \frac{1}{2\gamma} \lVert z - (x_0 - \gamma\nabla l(x_0)) \rVert_2^2\right]\\
u&=x_0-\gamma\nabla l(x_0)\\
\hat{x}&=\textrm{prox}_\gamma \phi(u)\\
\end{align*}



\subsection{(B)}
Now, we want to play around with our results to cast the lasso regression into a proximal gradient problem.
\begin{align*}
\hat{\beta}&=\arg\min_\beta \left\{\lVert y-X\beta\rVert_2^2 + \lambda \lVert \beta\rVert_1\right\}\\
l(\beta | X,y)&=\lVert y-X\beta\rVert_2^2=y^\top y - 2y^\top X\beta + \beta^\top X^\top X \beta\\
\hat{\beta}&=\arg\min_\beta \left\{l(\beta | X,y)+ \lambda \lVert \beta\rVert_1\right\}\\
l(\beta | X,y)\approx\hat{l}(\beta | X,y ; \beta_0)&= l(\beta_0 | X, y) + (\beta-\beta_0)^\top \nabla l(\beta_0 | X, y)\\
\nabla l(\beta | X, y)&=0 -2X^\top y + 2X^\top X\beta\\
\hat{l}(\beta | X,y ; \beta_0)&=\lVert y-X\beta_0\rVert_2^2 + (\beta-\beta_0)^\top \left(-2X^\top y + 2X^\top X\beta_0\right)
\end{align*}
Now, in the linear approximation to $l(\beta | X,y)$, we add in the regularization:
\begin{align*}
\tilde{l}(\beta | X,y ; \beta_0)&=\lVert y-X\beta_0\rVert_2^2 + (\beta-\beta_0)^\top \left(-2X^\top y + 2X^\top X\beta_0\right)+\frac{1}{2\gamma}\lVert \beta-\beta_0\rVert_2^2
\end{align*}
Now, we let $l(\beta|X, y)\approx \tilde l(\beta | X,y ;\beta_0)$ when $\beta$ is near $\beta_0$.
This is now exactly the form of surrogate optimization referenced above so
\begin{align*}
\phi(\beta)&=\lambda \lVert \beta \rVert_1\\
u^{(t)}&=\beta^{(t)}-\gamma^{(t)} \nabla l(\beta^{(t)} | X, y)= \beta^{(t)} - \gamma^{(t)}\left(2X^\top X\beta^{(t)}-2X^\top y\right)\\
\beta^{(t+1)}&=\textrm{prox}_{\gamma^{(t)}} \phi(u^{(t)})\\
\beta_i^{(t+1)}&=S_{\lambda\gamma^{(t)}}\left(u_i^{(t)}\right)=\textrm{sign}\left(u_i^{(t)}\right)\left(\left|u_i^{(t)}\right|-\lambda\gamma^{(t)}\right)_+
\end{align*}
So, to go from step $t$ to step $t+1$, we just compute $u^{(t)}$ then use its components to compute $\beta_i^{(t+1)}$.

There's a relatively high one-time cost to compute $X^\top X$ and $X^\top y$, and (depending on how big $p$, the number of elements of $\beta$, is) this cost carries over each iteration to compute $X^\top X \beta^{(t)}$. That's a $O(p^2)$ calculation (at least in the dense case). Beyond that, the rest of the operations are $O(p)$.








\end{document}